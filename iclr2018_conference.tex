\documentclass{article} % For LaTeX2e
\usepackage{iclr2018_conference,times}
\usepackage{hyperref}
\usepackage{url}


\title{(tentative) Learning Neural Markers of Schizophrenia Disorder Using Recurrent Neural Networks}

% Authors must not appear in the submitted version. They should be hidden
% as long as the \iclrfinalcopy macro remains commented out below.
% Non-anonymous submissions will be rejected without review.

\author{Antiquus S.~Hippocampus, Natalia Cerebro \& Amelie P. Amygdale \thanks{ Use footnote for providing further information
about author (webpage, alternative address)---\emph{not} for acknowledging
funding agencies.  Funding acknowledgements go at the end of the paper.} \\
Department of Computer Science\\
Cranberry-Lemon University\\
Pittsburgh, PA 15213, USA \\
\texttt{\{hippo,brain,jen\}@cs.cranberry-lemon.edu} \\
\And
Ji Q. Ren \& Yevgeny LeNet \\
Department of Computational Neuroscience \\
University of the Witwatersrand \\
Joburg, South Africa \\
\texttt{\{robot,net\}@wits.ac.za} \\
\AND
Coauthor \\
Affiliation \\
Address \\
\texttt{email}
}

% The \author macro works with any number of authors. There are two commands
% used to separate the names and addresses of multiple authors: \And and \AND.
%
% Using \And between authors leaves it to \LaTeX{} to determine where to break
% the lines. Using \AND forces a linebreak at that point. So, if \LaTeX{}
% puts 3 of 4 authors names on the first line, and the last on the second
% line, try using \AND instead of \And before the third author name.

\newcommand{\fix}{\marginpar{FIX}}
\newcommand{\new}{\marginpar{NEW}}

%\iclrfinalcopy % Uncomment for camera-ready version, but NOT for submission.

\begin{document}


\maketitle

% Outline: 
% 1. Introduction: RNNs and CNNs being used for variety of recognition and diagnosis tasks. Problem with previous attempts
% to learn features from brain imaging data (fmri).  
% 2. Related Work: previous work on frmi, previous work to learn from videos, previous work to diagnose schizophrenia 
% 4. Methods
% 	  4.1 dataset
%   4.2 recurrent-convolutional neural nets for feature learning from fmri segments
% 5. Experiments: 
%   5.1 effectiveness of RNNs and R-CNNs to learn discriminative features from fmri
%	  5.2 generalizability of learned features across datasets
% 6. Discussion
% 7. Conclusion

\begin{abstract}
... Here we propose a new method based on recurrent-convolutional neural networks to automatically learn useful representations from short segments of fMRI recordings...
\end{abstract}

\section{Methods}


\subsection{Style}


\subsubsection*{Acknowledgments}

Use unnumbered third level headings for the acknowledgments. All
acknowledgments, including those to funding agencies, go at the end of the paper.

\bibliography{iclr2018_conference}
\bibliographystyle{iclr2018_conference}

\end{document}
