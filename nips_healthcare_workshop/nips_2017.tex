\documentclass{article}

% if you need to pass options to natbib, use, e.g.:
% \PassOptionsToPackage{numbers, compress}{natbib}
% before loading nips_2017
%
% to avoid loading the natbib package, add option nonatbib:
% \usepackage[nonatbib]{nips_2017}

\usepackage[nonatbib]{nips_2017}
\usepackage[numbers]{natbib}
%\usepackage[backend=bibtex,style=ieee-alphabetic,natbib=true]{biblatex}
%\addbibresource{IEEEfull.bib}
%\addbibresource{IEEEexample.bib}

% to compile a camera-ready version, add the [final] option, e.g.:
% \usepackage[final]{nips_2017}

\usepackage[utf8]{inputenc} % allow utf-8 input
\usepackage[T1]{fontenc}    % use 8-bit T1 fonts
\usepackage{hyperref}       % hyperlinks
\usepackage{url}            % simple URL typesetting
\usepackage{booktabs}       % professional-quality tables
\usepackage{amsfonts}       % blackboard math symbols
\usepackage{nicefrac}       % compact symbols for 1/2, etc.
\usepackage{microtype}      % microtypography
\usepackage{graphicx}


\title{Learning Neural Markers of Schizophrenia Disorder Using Recurrent Neural Networks}
% what about this (or else ..): Schizophrenia Identification from 4-D fMRI data Using Recurrent Neural Networks

\author{
  David S.~Hippocampus\thanks{Use footnote for providing further
    information about author (webpage, alternative
    address)---\emph{not} for acknowledging funding agencies.} \\
  Department of Computer Science\\
  Cranberry-Lemon University\\
  Pittsburgh, PA 15213 \\
  \texttt{hippo@cs.cranberry-lemon.edu} \\
  %% examples of more authors
  %% \And
  %% Coauthor \\
  %% Affiliation \\
  %% Address \\
  %% \texttt{email} \\
  %% \AND
  %% Coauthor \\
  %% Affiliation \\
  %% Address \\
  %% \texttt{email} \\
  %% \And
  %% Coauthor \\
  %% Affiliation \\
  %% Address \\
  %% \texttt{email} \\
  %% \And
  %% Coauthor \\
  %% Affiliation \\
  %% Address \\
  %% \texttt{email} \\
}

\begin{document}
% \nipsfinalcopy is no longer used

\maketitle

\begin{abstract}

%A key challenge in dealing with neuroimaging data comes from the inter- and intra-subject variability. A prominent cause of these variabilities are due to small differences in the functional cortical mapping between individuals which calls for neural markers that are position and scale invariant. Most previous machine learning studies examine these markers using hand-designed features, which collapses the time dimension into a single number. Yet such approach does not adhere to the spatial location of these markers. Here we propose a new method based on recurrent-convolutional neural networks to automatically learn useful representations from short segments of fMRI recordings. Our goal is to exploit both spatial and temporal information in the fMRI movie (at the whole-brain voxel level) for identifying patients with schizophrenia.
Building smart systems that can accurately diagnose patients with mental disorders and identify effective treatments based on brain functional imaging data are of great applicability and are gaining much attention. Most previous machine learning studies use hand-designed features, such as functional connectivity, which does not maintain the potential useful information in the spatial relationship between brain regions and the temporal profile of the signal in each region. Here we propose a new method based on recurrent-convolutional neural networks to automatically learn useful representations from segments of 4-D fMRI recordings. Our goal is to exploit both spatial and temporal information in the functional MRI \textit{movie} (at the whole-brain voxel level) for identifying patients with schizophrenia.    


\end{abstract}

\section{Introduction}

Diagnosis of psychiatric diseases can be a prolonged process. This is mainly due to lack of objective biological markers (such as the level of a metabolite in a patient's blood test) associated with various mental disorders. Delayed and sometime inaccurate diagnosis, which might happen due to similarity of symptoms among different diseases (e.g. depression phase of bipolar disorder and unipolar depression) results in belated less effective intervention. There is also no objective biological marker for predicting treatment response in an individual. This oftentimes results in multiple changes in a patients prescription, resulting in poor adherence given the medications side effects. Such inefficiency in diagnosis and treatment prognosis process in psychiatric disorders is reflected in the global burden of disease, with mental illness ranking first, before cancer and cardiac conditions, in terms of time lost to disability (WHO 2012 report) and costs \citep{Roehrig2016}.

In recent years, machine learning techniques have shown success in identifying patients with mental or neurological disorders or predicting treatment response using brain imaging, especially structural and functional MRI (magnetic resonance imaging) data \citep{Orru2012, Zarogianni2013, Koutsouleris2016, Vieira2017, Gheiratmand2017}. Almost all these studies extract features from imaging data and use the features in combination with standard classifiers, such as support vector machines (SVM) \citep{Orru2012, Wolfers2015} to discriminate between patients and controls or predict response to treatment. (Less studies are available on the treatment response prediction, compared to diagnosis, mainly due to lack of data in the former field.) (Some typical imaging features extracted from functional MRI (fMRI) or structural MRI (sMRI) data include functional connectivity (FC), amplitude of low-frequency fluctuations (ALFF) for fMRI, and voxel-based morphometry and gray matter thickness/volume for sMRI. Such features may be extracted voxel-wise (where every voxel is a brain tissue of size $ \sim 1-27 mm^3$) or region-wise, from predefined brain regions (such as thalamus, postcentral gyrus, etc.).

With deep learning techniques gaining outstanding performance in various fields including image classification, speech recognition, and video classification, among others, the approach is being recruited and explored in clinical applications, including those involving medical imaging data \citep{Shen2017, Litjens2017, Gulshan2016}. In addition to their potential to surpass the performance of other standard machine learning techniques, deep learning  methods are attractive because they can be applied directly to the data, skipping the need to extract hand-designed features – a step that is necessary in almost all other machine learning approaches. In addition to the possibility of improving prediction accuracies, deep neural networks (DNN) have the potential to learn representations from brain images that might reveal useful information about abnormalities associated with the disease.

Various deep learning methods have been employed in analysis of imaging data for various psychiatric and neurological disorders, including but not limited to Alzheimer’s disease, ADHD, and Psychosis \citep[see][for a review]{Vieira2017}. These methods include multi-layer perceptron, autoencoders, deep belief networks, and convolutional neural networks. Most of these studies use sMRI for predictions in neurological disorders, and much fewer studies use fMRI \citep{Plis2014, Kim2016, Suk2016, Sarraf2016}, which has been shown to be particularly relevant in predictive analysis of psychiatric disorders (such as schizophrenia) (e.g. \citep{Damaraju2014, Calhoun2009}). fMRI data measures blood oxygenation level-dependent (BOLD) signal at every brain voxel by taking a scan of the whole brain every 1-3 s. It can be thought of as a movie of the brain activity (reflected in BOLD signal – although the relationship between BOLD signal and neural activity is still under scrutiny \citep{RN1}), either in response to a task (e.g. a motor, sensory, or cognitive task) or simply at rest.
%\section{Related Work}

%Various deep learning methods have been employed in analysis of imaging data for various psychiatric and neurological disorders, including but not limited to Alzheimer’s disease (AD), ADHD, and Psychosis \citep[see][for a review]{Vieira2017}. These methods include multi-layer perceptron (MLP), autoencoders (AE), deep belief networks, and convolutional neural networks. Most of these studies use sMRI for predictions in neurological disorders, and much fewer studies use fMRI, which has been shown to be particularly relevant in predictive analysis of psychiatric disorders (such as schizophrenia) (e.g. \citep{Damaraju2014, Calhoun2009}). fMRI data measures blood oxygenation level-dependent (BOLD) signal at every brain voxel by taking a scan of the whole brain every 1-3 s. It can be thought of as a movie of the brain activity (reflected in BOLD signal – although the relationship between BOLD signal and neural activity is still under scrutiny \citep{RN1}), either in response to a task (e.g. a motor, sensory, or cognitive task) or simply at rest.

%\citet{Plis2014} showed that a restricted Boltzman machine (RBM) \citep{Hinton2002} can extract features from fMRI data, that are comparable to spatial and temporal maps extracted by ICA (independent component analysis) analysis – one of the most commonly used methods to extract brain functional networks from fMRI data. They also showed that increasing the depth of a Deep Belief Network (DBN) \citep{Hinton2006} (from 1 to 3), generated by stacking RBM blocks, applied to sMRI data from patients with schizophrenia and healthy controls, results in features that discriminate more accurately between the two groups (from and F-score of 0.66 to 0.9, resp., in combination with an rbf-SVM). \citet{Suk2016} used a combination of a deep auto-encoder (DAE) and a hidden Markov Model (HMM) to discriminate between patients with mild cognitive impairment (MCI) and healthy controls. (They also modelled the dynamics of the FC networks in resting-state fMRI, assuming FC networks change over time and are not static.) First a DAE finds an embedding of the functional relations between brain regions (similar to \citep{Plis2014}), then a HMM estimates the dynamic characteristics of the functional networks in the embedded space for the MCI and control group separately. During testing, the new sample is assigned to the group (MCI vs. HC) with the higher likelihood probability. \citet{Kim2016} used FC features, extracted from resting-state fMRI, in combination with a sparse autoencoder-based pretrained DNN, to discriminate between patients with schizophrenia and healthy controls. The DNN's accuracy (85.5\%) surpassed the rbf SVM accuracy on the same dataset by 8.1\%. \citet{Sarraf2016} used 2-D convolutional neural network (CNN) to extract features from fMRI (and MRI) data for discriminating between patients with Alzheimer’s disease and healthy controls. In their approach, they decomposed each (resting-state) 4D fMRI data into a series of 2-D images along z-axis (axial slices) and time and fed those as input to CNN architectures, LeNet \citep{Lecun1998} and GoogleNet \citep{Szegedy2015}, to train and test the adopted networks. (Authors achieved a very high classification accuracy of almost 100\% on the test set (as well as average over cross-validation folds), but it is not clear whether the set of subjects in the train and test sets, as well as cross-validation folds, are disjoint. This is extremely important, as we will describe in our own work later in the Methods, to avoid \textit{double-dipping}. If the samples from one subject appear in both train and test sets, there is no surprise the results are extremely good.)

Here, our goal was to exploit both spatial and temporal information in the fMRI movie (at the whole-brain voxel level) for identifying patients with schizophrenia. We propose using a recurrent convolutional neural network (R-CNN). The proposed model involves a 3-D CNN followed by a sequential NN with LSTM units. The CNN extracts spatial features, which are fed to the LSTM model, which uses the dependencies between time points at every spatial location to generate an output \{patient, control\} (see Figure-\ref{fig1}). To our knowledge, this is the first work to apply a recurrent CNN to fMRI data for neurological/psychiatric diagnosis (here schizophrenia). As discussed earlier, most previous fMRI/machine learning studies, including some deep learning ones \citep{Kim2016}, use hand-designed features, in particular FC features \citep{Gheiratmand2017}, which collapses the time dimension (of the BOLD signal) into one single number (i.e. the correlation coefficient between a pair of time-series). Such approach does not keep track of the relationships between spatial locations (e.g. voxel or brain regions) either. Here, we expand the work by \citet{Bashivan2016}, who successfully applied a R-CNN (with 2-D convolutions) to EEG data in a mental load classification task, to fMRI data (using 3-D convolutions). %[[video classification idea]] 
We used fMRI data in response to an auditory oddball task from patients diagnosed with schizophrenia and healthy controls from FBIRN dataset \citep{Keator2016}. The task is to predict a label \{patient, control\} based on the preprocessed fMRI BOLD signal at the voxel level, exploiting the temporal and spatial information in the data within an end-to-end deep learning framework.

\section{Methods}

\subsection{Dataset}

We used FBIRN phaseII fMRI dataset \citep{Keator2016}.
%, downloaded from Function BIRN Data Repository, Project Accession Number 2007-BDR-6UHZ1 (fbirnbdr.nbirn.net:8080/BDR). 
It includes functional and structural MRI data for patients with schizophrenia or schizoaffective disorder and age- and sex-matched healthy controls. We focused on the subset of fMRI data acquired in response to an Auditory Oddball (AO) task. 
%Why did we choose this dataset??
The fMRI data included whole-brain scans taken every 2 seconds for a period of 280 seconds. Per subject, there were 4 experiment runs (280 sec each).
A standard series of preprocessing stages was applied to each subject's raw fMRI data using the FSL software package \citep{Jenkinson2012}. These included motion correction, tCompCor denoising, spatial filtering, high-pass temporal filtering, and linear registration to the MNI T1 template through subject's T1 scan \citep[see][for a more detailed description of the preprocessing stages]{Gheiratmand2017}. The first 3 volumes in each run were deleted for signal instability, resulting in a total of 137 volumes. Finally, a universal mask, i.e. the intersection of all subjects' brains, was applied to each subject's data, resulting in a common non-zero brain area comprising 26,949 voxels (brain tissue of size $3.4374\times3.4375\times5$ mm).
After preprocessing and quality control, N = 95 subjects (46 patients, 49 controls) remained in the study (from a total of 164 subjects in both scanning sessions). (We used the data from the second scanning session, out of two, for no particular reason.)

\subsection{Recurrent-Convolutional Neural Networks}

We compared accuracy of recurrent neural networks (LSTM model) and recurrent-convolutional neural networks (R-CNN) with baseline classifiers to learn discriminative features for the problem of distinguishing between patients with schizophrenia and healthy individuals. We tried several different architectures consisting of LSTMs as well as R-CNNs. The input data was structured as a 3-dimensional tensor (3-D brain scans). 

\textbf{Baseline Classifier}: We compared our proposed models with the results from a linear support vector machine which we call the baseline method. In order to make the SVM solution feasible, we reduced the size of the input by a rate of $4\times4\times3$ voxels. This resulted in a feature vector of size 77,953 (137 time points $\times$ 569 "supervoxels") per run per subject, for a total of 380 samples (95 subjects $\times$ 4 runs).  

\textbf{LSTM model}: All voxel values are reshaped into a vector and fed directly into a two-layer forward LSTM model. LSTMs are known for their ability in learning long-term dependencies between inputs. In this model the spatial relationship between voxels are ignored and LSTM learns the temporal relationship between activations in different voxels. 

\textbf{Conv-LSTM model}: 3-D activations for each time frame were fed into a 3-D CNN. We used 3-dimensional convolutions followed by max-pooling to extract position and scale independent features that would generalize across individuals. Identical networks were used to process each time-frame with shared weights between them. Outputs of each CNN were then reshaped into a vector that was fed as input to the LSTM network at each time step. Figure-\ref{fig1} shows an overview of the recurrent-convolutional network used here. The CNN part learns 3-D features that are invariant to translation and scaling and reduces the dimension of the input space before feeding it into LSTM. ?? filters were used in convolutions in each layer. 

\subsection{Training Details}
\label{training_details}

We used Adam optimizer with default parameter settings ($\beta_1=0.9, \beta_2=0.999, \epsilon=10^{-8}$). We included dropout in the input and outputs of the LSTM cells \citep{Zaremba2014} which contained most of the tunable parameters in our models. 
10-fold cross-validation was used to evaluate the performance for each proposed model. $l_2$ regularizer was used on all convolutional and fully connected weights. 
3-D-convolutions of size $3\times3\times3$ with stride 1 were used. Each block of convolutions were followed by a maxpool layer of size 2 and stride 2. Feature map size was divided by 2 after each maxpool layer. Size of the input to the network was $(53\times64\times37)$. 
A batch size of 64 was used for all models. Larger convolutional models were trained synchronously on 16 GPUs. In both cases we experimented with varying number of time-frames included in each sample (i.e. 16 and 64).
A different subset of subjects were assigned to the test set for each fold and data from the rest of the subjects were used for training and validation (without overlap between train and validation sets). For training, samples within each batch were generated by randomly selecting time windows from different subjects. All possible time windows of size $T$ from all subjects were used for training, validation and test. For each fold, training was done for 10 folds. We used the validation set for early stopping of training. After each training epoch, performance on the validation set was computed. Training was performed for 10 epochs and test performance was computed for the network state with highest validation score. 
%experimental setup: 

%We performed all experiments on the XSEDE Xstream GPU cluster \citep{Towns2014} -- a petaFLOP Cray machine with 8 NVIDIA K80 card, 2 Intel Ivy-Bridge 10-core CPUs, and 256 GB of DRAM per node, with shared Lustre file system.


\begin{figure}[t]
\begin{center}
\includegraphics[width=2.5in]{figures/overview.png}
\end{center}
\caption{Overview of the method used.}
\label{fig1}
\end{figure}

\subsection{Data Preparation}

Preprocessed fMRI data was normalized in three steps: For each subject/run/voxel (i) the BOLD signal time-series was demeaned; (ii) The resulting time-series was divided by standard deviation (SD) of activation across the whole brain (all voxels and times). Next, each voxel's time-series was standardized respective to the mean and SD of the voxel's activation across all subjects and runs.

Each fMRI sample (subject/run) was split into windows of 64 time points (or 128s) (i.e. a total of 137-63=74 samples), which formed the input for training, evaluation, and testing of our models. A shorter time-frame of 16 volumes (32s) was also tested. Subjects in the training, evaluation, and test sets were disjoint, i.e. all samples corresponding to a subject were used only in one of the three sets.

\section*{Results}

We investigated the effectiveness of LSTM and R-CNNs in capturing the temporal and spatial structure of the fMRI data in order to distinguish between patients and controls. The average accuracy of the baseline method was 57.89\%. 
Both LSTM and R-CNN models performed better than the baseline model (Table \ref{table_1}). Our best LSTM model performed slightly better than our R-CNN model ($ \sim 1\%$). Additionally, we explored the impact of length of the samples on the performance of the trained network . We found that larger time windows improved the accuracy in all tested models. For the R-CNN models we experimented with the number of convolutional layers and found that the deeper CNN models did not reduce the error (and slightly increased it). 

% \begin{table}[h]
% \centering
% \caption{Comparison of different model architectures and time-windows on the AO dataset.}
% \label{table_1}
% \begin{tabular}{|c|c|c|c|c|c|c|c|}
% \hline
% \textbf{Model}           & \multicolumn{2}{c|}{\textbf{LSTM}} &
% \multicolumn{2}{c|}{\textbf{RCNN (2, 1)}} & \multicolumn{2}{c|}{\textbf{RCNN (1, 2)}} & \textbf{RCNN (2, 2, 1)} \\ \hline
% \textbf{Window Size}      & 16   & 64  & 16  & 64    &16    & \ 64    & 64           \\ \hline
% \textbf{Test Performance} & 60.2 \%   & 66.4\%   & 63.1\% & 64.9\% & 60.9\%  & 61.4\%  & 63.3\%  \\  \hline
% \end{tabular}
% \end{table}

\begin{table}[h]
\centering
\caption{Comparison of different model architectures and time-windows on the AO dataset.}
\label{table_1}
\begin{tabular}{ c c c c c c c c }
\hline
\textbf{Model}           & \multicolumn{2}{c}{\textbf{LSTM}} &
\multicolumn{2}{c}{\textbf{RCNN (2, 1)}} & \multicolumn{2}{c}{\textbf{RCNN (1, 2)}} & \textbf{RCNN (2, 2, 1)} \\ \hline
\textbf{Window Size}      & 16   & 64  & 16  & 64    &16    & \ 64    & 64           \\ \hline
\textbf{Test Performance} & 60.2 \%   & 66.4\%   & 63.1\% & 64.9\% & 60.9\%  & 61.4\%  & 63.3\%  \\  \hline
\end{tabular}
\end{table}

% Do we have this table??
%Table 2 shows the number of parameters used by each type of architecture, and the corresponding error achieved on the test set.

The best recurrent convolutional model was obtained with architecture \#2 containing two back-to-back convolutions in the first layer, one convolution in the second layer, and two back-to-back LSTMs in the third layer. We compared results using both 16 and 64 time-window samples while keeping the batch size fixed at 64 samples. For the model using 64 time windows, we noticed a significant improvement of 1.8\% in classification accuracy. For the LSTM model, the improvement with increasing the time window was over 6\%. Comparing the performance of the baseline model and deep learning, the test scores of LSTM and R-CNN were $\sim$ 8\% better than linear SVM.

\section{Discussion}
%A key challenge in dealing with neuroimaging data comes from the inter- and intra-subject variability that any reliable diagnosis system should be robust to. A prominent source of these variabilities is due to small differences in the functional cortical mapping between individuals which calls for neural markers that are position and scale invariant. We applied several neural network architectures to learn invariant markers for schizophrenia disorder from a large-scale fMRI dataset. While both of these methods achieved remarkable accuracy in distinguishing between schizophrenic and control subjects, they fell short of reaching the same level as hand-designed connectivity features \citep{Gheiratmand2017}. The 2-layer LSTM architecture which was used in all our models helped to learn the temporal dependencies in long time windows. The R-CNN model learned better representations when using short time windows ($T=16$) but failed to match the LSTM performance when using the longer time windows ($T=64$). 
%The 3-D convolutions formed more stable features compared to the LSTM model which was evident in significantly smaller variance in performance.

We applied several neural network architectures to learn invariant markers for schizophrenia disorder from a large-scale fMRI dataset. In particular we tried sequential LSTM networks alone and in combination with CNN input layers. While both of these methods achieved remarkable accuracy in distinguishing between patients with schizophrenia and healthy control subjects, they fell short of reaching the same level as hand-designed connectivity features \citep{Gheiratmand2017}. This may be due to the relatively small sample size for an end-to-end deep learning framework. The 2-layer LSTM architecture which was used in all our models helped to learn the temporal dependencies in long time windows. The R-CNN model learned better representations compared to the LSTM model when using short time windows ($T=16$) but failed to match the LSTM performance when using the longer time windows ($T=64$). In future work, we will exploit other fMRI data subsets in the FBIRN dataset that are acquired in response to three other tasks (working memory, sensorimotor, and breath hold) to enrich the sample space. Applying transformations (such as wavelets) to fMRI data before feeding to DNN might also help improve the subtle signal in the data, compared to directly feeding the BOLD signal to the network.  

%\subsubsection*{Acknowledgments}

%Data used for this study were downloaded from the Function BIRN Data Repository (http://fbirnbdr.birncommunity.org:8080/BDR/), supported by grants to the Function BIRN (U24-RR021992) Testbed funded by the National Center for Research Resources at the National Institutes of Health, U.S.A. This work used the Extreme Science and Engineering Discovery Environment (XSEDE) XStream at through allocation TG-MCB090174. We acknowledge funding support from the Graduate Assistance in Areas of National Need (GAANN) fellowship and the NSF Award 1443054 (SPIDAL). 

\bibliography{nips_2017}
\bibliographystyle{ieeetr}
%\printbibliography

\end{document}
